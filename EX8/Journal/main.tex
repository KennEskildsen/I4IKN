%----------------------------------------------------------------------------------------
%	PACKAGES AND DOCUMENT CONFIGURATIONS
%----------------------------------------------------------------------------------------

\documentclass{article}

\usepackage{graphicx} 										% Inkluder for billeder
\usepackage{amsmath} 										% Inkluder for Matematiske operatorer
\usepackage{listings}             							% Inkluder for src 
\usepackage{color}											% Til senere brug ved src
%sexy pdf'er
\usepackage{pdfpages}
\usepackage{pdflscape} 
\usepackage[utf8]{inputenc}									% Danske udtryk (fx figur og tabel) samt dansk orddeling og fonte med
\usepackage[danish]{babel}									% danske tegn. Hvis LaTeX brokker sig over æ, ø og å skal du udskifte
\usepackage[T1]{fontenc}									% "utf8" med "latin1" eller "applemac". 
\usepackage{mflogo}

\setlength\parindent{0pt} 									% Fjerner indryk for alle paragraphs

\renewcommand{\labelenumi}{\alph{enumi}.} 					% Sætter nummering i "enumerate" environment til bogstaver 
															% i stedet for tal

% håndtering af src
%Til programkode
\usepackage{listings}
\usepackage{color}

\definecolor{dkgreen}{rgb}{0,0.6,0}
\definecolor{gray}{rgb}{0.5,0.5,0.5}
\definecolor{mauve}{rgb}{0.58,0,0.82}
 
\lstset{ 
  language=C++,                								% the language of the code
  basicstyle=\footnotesize,      					  		% the size of the fonts that are used for the code
  numbers=left,                   							% where to put the line-numbers
  numberstyle=\tiny\color{gray},  							% the style that is used for the line-numbers
  stepnumber=1,                   							% the step between two line-numbers. If it's 1, each line will be numbered
  numbersep=5pt,                  							% how far the line-numbers are from the code
  backgroundcolor=\color{white},  							% choose the background color. You must add \usepackage{color}
  showspaces=false,               							% show spaces adding particular underscores
  showstringspaces=false,         							% underline spaces within strings
  showtabs=false,                 							% show tabs within strings adding particular underscores
  frame=single,                   							% adds a frame around the code
  rulecolor=\color{black},        							% if not set, the frame-color may be changed on line-breaks within not-black text (e.g. commens (green here))
  tabsize=2,                      							% sets default tabsize to 2 spaces
  captionpos=b,                   							% sets the caption-position to bottom
  breaklines=true,                							% sets automatic line breaking
  breakatwhitespace=false,        							% sets if automatic breaks should only happen at whitespace
  title=\lstname,                   						% show the filename of files included with \lstinputlisting;
                                  							% also try caption instead of title
  keywordstyle=\color{blue},       							% keyword style
  commentstyle=\color{dkgreen},     						% comment style
  stringstyle=\color{mauve},        						% string literal style
  escapeinside={\%*}{*)},           						% if you want to add LaTeX within your code
  morekeywords={*,...},             						% if you want to add more keywords to the set
  rangeprefix=//----------,									% Used for sexy code includes
  rangesuffix=----------,									% ---||---
  includerangemarker=false,									% ---||---
  literate=
  {á}{{\'a}}1 {é}{{\'e}}1 {í}{{\'i}}1 {ó}{{\'o}}1 {ú}{{\'u}}1
  {Á}{{\'A}}1 {É}{{\'E}}1 {Í}{{\'I}}1 {Ó}{{\'O}}1 {Ú}{{\'U}}1
  {à}{{\`a}}1 {è}{{\`e}}1 {ì}{{\`i}}1 {ò}{{\`o}}1 {ù}{{\`u}}1
  {À}{{\`A}}1 {È}{{\'E}}1 {Ì}{{\`I}}1 {Ò}{{\`O}}1 {Ù}{{\`U}}1
  {ä}{{\"a}}1 {ë}{{\"e}}1 {ï}{{\"i}}1 {ö}{{\"o}}1 {ü}{{\"u}}1
  {Ä}{{\"A}}1 {Ë}{{\"E}}1 {Ï}{{\"I}}1 {Ö}{{\"O}}1 {Ü}{{\"U}}1
  {â}{{\^a}}1 {ê}{{\^e}}1 {î}{{\^i}}1 {ô}{{\^o}}1 {û}{{\^u}}1
  {Â}{{\^A}}1 {Ê}{{\^E}}1 {Î}{{\^I}}1 {Ô}{{\^O}}1 {Û}{{\^U}}1
  {œ}{{\oe}}1 {Œ}{{\OE}}1 {æ}{{\ae}}1 {Æ}{{\AE}}1 {ß}{{\ss}}1
  {ç}{{\c c}}1 {Ç}{{\c C}}1 {ø}{{\o}}1 {å}{{\r a}}1 {Å}{{\r A}}1
  {€}{{\EUR}}1 {£}{{\pounds}}1
}






%----------------------------------------------------------------------------------------
%	DOKUMENT INFORMATION
%----------------------------------------------------------------------------------------

\begin{document}
\begin{titlepage}
\begin{center}
\line(1,0){300}\\
[0.25in]
\huge{SOCKET PROGRAMMERING} 											% Titel
\line(1,0){300}\\
[1cm]
\huge{I4IKN\\ Journal Øvelse 8+9} \\
Gruppe 8\\ %TODO 														% Gruppe nr.
[3cm]

\large
\begin{tabular}{l l l}
Afleveres: 		& Mandag 26 Oktober 2015 & 						\\ 		% Dato for aflevering
																\\
Udarbejdet af:  & Karsten Schou Nielsen  & Stud.nr.: 201370045  \\
				& Kasper Behrendt		 & Stud.nr.: 20071526   \\
				& Kenn H Eskildsen 		 & Stud.nr.: 201370904  \\
																\\
Underviser:		& Torben Gregersen 								\\		% Underviser

\end{tabular}
\end{center}
\end{titlepage}

%----------------------------------------------------------------------------------------
%	TOC
%----------------------------------------------------------------------------------------
\newpage
\tableofcontents
\thispagestyle{empty}
\cleardoublepage
\setcounter{page}{1}
%----------------------------------------------------------------------------------------
%	SEKTION 1
%----------------------------------------------------------------------------------------
\newpage
\section{Formål}

Skriv en iterativ TCP-server med support for en client ad gangen, som kan modtage
en tekststreng fra en client. Serveren skal køre i en virtuel Linux-maskine.
Tekststrengen skal indeholde et filnavn + en eventuel stiangivelse. Tilsammen skal
informationen i tekststrengen udpege en fil af en vilkårlig type/størrelse beliggende i
serveren, som en tilsluttet client ønsker at hente fra serveren. Hvis filen ikke findes
skal serveren returnere en fejlmelding til client’en. Hvis filen findes skal den overføres
fra server til client i segmenter på 1000 bytes ad gangen – indtil filen er overført
fuldstændigt. Serverens portnummer skal være 9000. Server-applikationen skal
kunne startes fra en terminal med kommandoen.\\
%./file_server (for C/C++ applikationers vedkommende) %TODO 

%Den samlede sourceskode til file_server.cpp ses på Bilag 1. 

Herunder vil udvalgte dele af koden blive beskrevet.\\





\begin{lstlisting}
int sockfd, connfd, portno;
     socklen_t clilen;
     char buffer[256];						
     struct sockaddr_in serv_addr, cli_addr;
     int n;
     if (argc < 2) {
         fprintf(stderr,"ERROR, no port provided\n");
         exit(1);
     }
     sockfd = socket(AF_INET, SOCK_STREAM, 0);
     if (sockfd < 0) 
        error("ERROR opening socket");
        bzero((char *) &serv_addr, sizeof(serv_addr));
//atoi konverterer ascii to integer. 
     portno = atoi(argv[1]);
     serv_addr.sin_family = AF_INET;
     serv_addr.sin_addr.s_addr = INADDR_ANY;
     serv_addr.sin_port = htons(portno);
//bind binder Socket-adr til server-adr. Ved fejl er adressen sikkert allerede brugt 
     if (bind(sockfd, (struct sockaddr *) &serv_addr,
              sizeof(serv_addr)) < 0){ 
              close(sockfd);
              error("ERROR on binding");
     }

\end{lstlisting}  



%----------------------------------------------------------------------------------------
%	SEKTION 2
%----------------------------------------------------------------------------------------
\newpage
\section{Udførelse}




%----------------------------------------------------------------------------------------
%	SEKTION XX
%----------------------------------------------------------------------------------------
\newpage
\section{XX}

blablablabla


\end{document}